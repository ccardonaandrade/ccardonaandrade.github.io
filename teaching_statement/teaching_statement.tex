\documentclass{article}

\usepackage[utf8]{inputenc}
\usepackage[english]{babel}

\setlength{\parindent}{1em}
\setlength{\parskip}{1em}
\renewcommand{\baselinestretch}{1.3}

\usepackage{ragged2e}
\usepackage[margin=1.2in]{geometry}

\begin{document}

\centering
\textbf{Teaching Statement}

Carlos Cardona Andrade

University of Warwick

\justify

%My passion for teaching was sparked during my undergraduate years when I had the chance to teach business students Principles of Economics. This initial experience has evolved into a diverse and fulfilling teaching career. I have taught a wide range of courses, from highly technical subjects like Applied Econometrics and Calculus to more narrative-based classes such as Economics for Business. This variety challenges me to adapt my teaching style to different learning preferences, making me a more versatile and effective educator. My philosophy emphasizes adaptability, continuous learning, and the ability to make complex ideas engaging and relatable, regardless of the subject or audience.

My passion for teaching was sparked during my undergraduate years when I had the chance to teach business students Principles of Economics. This initial experience has evolved into a diverse and fulfilling teaching career, instructing courses in both English and Spanish. I have taught a wide range of subjects, from highly technical ones like Applied Econometrics and Calculus to more narrative-based classes such as Economics for Business. This variety challenges me to adapt my teaching style to different learning preferences, making me a more versatile and effective educator. My philosophy emphasizes adaptability, continuous learning, and the ability to make complex ideas engaging and relatable, regardless of the subject, audience, or language of instruction.

\medskip


A central tenet of my teaching approach is understanding my audience. Having taught students across disciplines like economics, political science, sociology, and business – as well as diverse nationalities and academic levels – I have learned to tailor my instruction to their unique perspectives and learning needs. When teaching quantitative methods to non-economics majors, student feedback prompted me to replace heavy abstraction with real-world examples and contextual applications relevant to their fields. Moreover, I strive to create an inclusive environment by acknowledging varying prior knowledge and cultural contexts. This adaptability allows students from all backgrounds to deeply engage with concepts and connect theory to empirics.




\medskip


A second guiding principle in my teaching is ensuring that students leave the classroom with concrete knowledge they can apply and build upon. To achieve this, I heavily incorporate visualizations, which I consider indispensable tools regardless of the topic. In math-intensive courses, I use graphs and charts to visually represent and explain statistical assumptions or the results of economic models, making the abstract more tangible. Similarly, in an economic history class, I would use timelines to reinforce connections between events, people, and institutions. Furthermore, I structure each lecture with intentionality, beginning with a roadmap that outlines the content and objectives, and concluding with a five-minute summary that reinforces the key takeaways. This structured approach, coupled with the strategic use of visuals, facilitates deeper understanding and ensures that students grasp the essential points, leaving them better equipped to apply the knowledge they have acquired.

\medskip

My teaching philosophy goes beyond simply imparting content knowledge. I see lectures as dynamic dialogues that encourage critical thinking and intellectual curiosity. To facilitate this, students prepare questions based on assigned readings for the next lecture. For instance, in the Quantitative Analysis course for sociologists, political scientists, and anthropologists, readings on quantification debates led to engaging discussions where students challenged perspectives using both the readings and their personal experiences. This approach has been very effective in promoting participation, ensuring that students are exposed to multiple viewpoints, and compelling them to actively engage with the material. During these discussions, I provide guidance, encouraging evidence-based reasoning while creating a respectful environment for diverse opinions. Ultimately, I believe true scholarship thrives when intellectual rigor and open-mindedness come together in an atmosphere of vigorous, thoughtful discourse, leading to a deeper understanding.

\medskip

While I have not directly instructed economic history courses, my engagement with the field has been substantial. During my doctoral studies at the University of Warwick, I took economic history as one of my optional modules, exposing me to the latest advancements in the field. Additionally, as a member of the economic history research group, I participated in weekly reading discussions analyzing seminal works by and contemporary papers from mainstream journals. These discussions, which included debates between historians and economists on developing economic history as a discipline within economics, fostered a deep appreciation for how it contextualizes broader economic concepts. I found the debate over whether there is a universal analytical framework for studying institutional change in history, akin to the use of supply and demand curves in economics to explain markets, particularly intriguing.

\medskip 


Teaching economic history presents a unique opportunity to cultivate global perspectives among students, allowing them to test their economic insights in the broader context of historical events and trends. By examining significant occurrences such as the Great Depression, the rise of globalization, or the Industrial Revolution, students gain valuable perspectives on major global issues of our time, including trade wars, financial crises, and migration pressures. Furthermore, exploring the history of economic thought promotes critical reasoning, as students evaluate debates across varying intellectual traditions and methodologies. Balancing theoretical frameworks with empirical evidence cultivates an essential proficiency: rendering informed judgments that synthesize conceptual insights and observable realities. Additionally, precise writing and storytelling are crucial and transferable skills that students can develop through these classes, enabling them to effectively communicate economic concepts and historical narratives with clarity and impact in any context.



\end{document}